
This course is an introduction to formal logic. The primary focus of the course is learning a formal language, which is then used for representing, explaining, and evaluating certain kinds of reasoning. There are several purposes for such a course. One is that it is an important sub-field of philosophy that is worth studying in it's own right. It also overlaps with mathematics and computer science, and it is an important part of those disciplines. (In fact, it is fundamental to computer science.) It is also applicable to any field where arguments and reasoning processes are analyzed and evaluated. But, beyond it's direct applications, learning logic is an important part of a complete education. Amanda Ripley makes the point plainly in \textit{The Smartest Kids in the World, And How They Got That Way}. Explaining the value of mastering ``the language of logic,'' she writes,

\begin{quote}
It is a disciplined, organized way of thinking. There is a right answer; there are rules that must be followed. More than any other subject, [it] is rigor distilled. Mastering the language of logic helps to embed higher-order habits in [our] minds: the ability to reason, for example, to detect patterns and to make informed guesses. Those kinds of skills have rising value in a world in which information is cheap and messy. (p.70)
\end{quote}

While formal logic is a part of philosophy, you will also find that it looks somewhat like math (and, as mentioned above, in some respects, it overlaps with math). As with mathematics, in formal logic, we apply formal techniques and use symbolic expression—for example, $P \lor Q \vdash \lnot P \rightarrow Q$. Don’t be intimidated by the symbols. The use of symbols makes the relevant characteristics of the sentences more obvious and manageable, and once you have a grasp of them, they present no difficulty.