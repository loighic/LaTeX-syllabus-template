%%%
%%%  Use XeLaTex !!!!!

\documentclass[11pt,oneside]{article}
\usepackage{syllabus}

% local definitions
\def\mydepartment{Department of Philosophy \& Religion}
\def\myaddress{George Hall}

\def\mytitle{\MakeUppercase{\Huge{Introduction to Philosophy}}}	
\def\subtitle{PHI 1103 (08)}				% Change section number.
\def\meeting{See Canvas for available meeting times.}
\def\semester{Spring 2021}					% Change semester.


\begin{document}

\vspace*{.75cm}
\begin{flushleft}
\textsf{\subtitle~~$\cdot$~~\semester\\
\vspace{.75cm}
%\subtitle\\
\vspace{3mm}
\mytitle\\
\vspace{5mm}
\normalsize{\myauthor}\\
\myemail\\}
%\myweb\\}
\vspace{1.5mm}
%\myphone~$\cdot$~\myaddress~$\cdot$~\meeting\\}
\end{flushleft}
\vspace{15mm}

\section{Preliminaries}

\noindent Email is the best way to reach me, and I will usually respond within 24 hours. If you would like a more immediate response, you can try calling or sending a text message to 267-416-0292. But don’t leave a voicemail. I won’t get it.
 
\vspace{2mm}
\noindent \meeting ~(\myaddress)

%With suitable notice, I may make any changes to the syllabus, readings, assignments, or schedule at any time during the semester.



\section{Readings}

Purchase the course pack for this section of Intro to Philosophy from Barnes \& Noble. It should also have my name (Johnson) on it somewhere.


\section{Course Description}

Before taking a philosophy course, most people are unfamiliar with the subject. So, by way of introduction, here is an excerpt from a lecture on Plato’s dialogue the \textit{Apology} (which we’ll read) by the philosopher Steven Smith, 
\begin{quote}
Philosophy grows out of a desire to replace opinion with knowledge, to replace opinion or belief with reason. For philosophy, it is not enough simply to hold a belief on faith, but one must be able to give a rational account, a reasoned account for one’s belief. Its goal, again, is to replace civic faith with rational knowledge. And, therefore, philosophy is necessarily at odds with belief and with this kind of civic faith. … The philosopher seeks to judge those beliefs in the light of true standards, in the light of what is always and everywhere true, as a quest for knowledge. There is a necessary and inevitable tension between philosophy and mere belief.
\end{quote}

Philosophy is different than science, but only to a degree. Science investigates the world by collecting data and doing experiments. Philosophy, meanwhile, usually undertakes neither of these activities. But still, the philosopher seeks knowledge, and the tool for seeking knowledge in this domain is reason. Given some facts (perhaps uncovered by science), what follows? What is likely to be---or must be---true? 

The theme of this course is challenging our beliefs. There are things that we want to believe, for instance, that we are real, that we have free will, that a god exists. How well do these beliefs hold up when we examine them and consider the evidence? We’ll see.

We will start out with a topic that doesn’t challenge our beliefs: arguments. These are the basic tool for doing philosophy. Next, we’ll consider Galileo. That the earth revolves around the sun shouldn’t challenge what you believe, but it sets the stage for the topics to follow. Why, in his day, was Galileo’s claim that the earth revolves around the sun threatening? What role did the evidence have in the debate about whether the sun revolves around the earth or the earth revolves around the sun? What else was a factor? 

After Galileo, we will examine the following questions. What is the mind and who has one? What does it take to be a person? Do we have free will? Are we living in a computer simulation? And does God exist? These are real questions, not philosophical exercises. Although interestingly, how these questions turn out doesn’t affect our day-to-day lives. When people finally accepted that the earth is not the center of the universe, nothing really changed (except for astronomers). Similarly, if it turns out that we don’t have free will or that we’re living in a computer simulation, you will still go on with your life. Nothing will suddenly seem different.

In the last three weeks of the semester, we will read two of Plato’s dialogues, the \textit{Apology} and the \textit{Crito}. The first is an account of the trial of Socrates. The ancient Greek philosopher Socrates was put on trial for, essentially, the very thing that this course is about, questioning commonly held beliefs and seeking knowledge. He was found guilty, and in the \textit{Crito} Socrates is waiting to be put to death. Here Plato examines the relationship between the individual citizen, especially one like Socrates, and the state.\\
%\begin{center}
%\symbol{\string"E0CA} \symbol{\string"E0C9} \symbol{\string"E0CA}		% The alphanumeric code has to match the font being used. 
%\end{center}

\noindent This is an online course, and it is not designed to be too similar to a classroom course. Instead
of lectures being the central feature of the course, you will be working independently on the
readings. I am available and will be in contact with you, and you should email me as often
as you find useful. But think of this course as a very structured independent study (that
is, structured by by me for you) rather than as a standard classroom course. See also the
additional information about the course in the first module in Canvas. 

% This is for Intro online and BE online.

\section{Learning Objectives}

Naturally, one objective is to become acquainted with the assigned material. Other, no less important, goals are improving reading comprehension, critical thinking, and writing skills.


\section{Schedule}

\begin{minipage}{\textwidth}
See the Google calendar in Canvas for the exact schedule. The calendar is also available \href{https://calendar.google.com/calendar/u/0?cid=YmoxZGY0djIxMmhiMjE1b3ZlMW1lbXFwdGtAZ3JvdXAuY2FsZW5kYXIuZ29vZ2xlLmNvbQ}{\url{HERE}}.

\vspace{3mm}
\begin{sffamily}
\begin{scriptsize}
%\begin{center}
%\begin{tabular}{@{}>{\columncolor{white}[0pt][\tabcolsep]}  l >{\columncolor{white}[\tabcolsep][0pt]} l @{}}
\begin{tabular}{l l}
\hline
\rowcolor{blue3}
\textbf{Weeks 1 \& 2} & {Arguments in philosophy}\\
\rowcolor{blue2}&\quad Johnson, ``An introduction to arguments''\\

\rowcolor{blue3}\textbf{Week 3} & {The earth is moving!??}\\
\rowcolor{blue2}&\quad Gingerich, ``The Galileo affair''\\ 

\rowcolor{blue3}\textbf{Weeks 4 \& 5} & {The mind}\\
\rowcolor{blue2}&\quad Johnson, ``What is the mind and who has one?''\\

\rowcolor{blue3}\textbf{Week 6} & {Being a person and end of life decisions}\\
\rowcolor{blue2}&\quad Arras, ``The severely demented, minimally functional patient''\\

\rowcolor{blue3}\textbf{Weeks 7 \& 8} & {Free will}\\
\rowcolor{blue2}&\quad Johnson, ``Could I have taken the other road? Libertarianism versus Determinism''\\ 
\rowcolor{blue2}&\quad Ayer, ``Freedom and necessity''\\ 

\rowcolor{blue3}\textbf{Week 9} & {Are we living in a computer simulation?}\\
\rowcolor{blue2}&\quad Bostrom, ``Why the probability that you are living in a matrix is quite high''\\

\rowcolor{blue3}\textbf{Week 10} & {Does God exist?}\\
\rowcolor{blue2}&\quad Rowe, ``God and evil''\\

\rowcolor{blue3}\textbf{Weeks 11 - 14} & {Socrates on trial}\\
\rowcolor{blue2}&\quad Plato, ``Apology'' and ``Crito''\\

%\rowcolor{blue3}Week 15 & no new reading\\
\hline
\end{tabular}
%\end{center}
\end{scriptsize}
\end{sffamily}

%\vspace{3mm}
\end{minipage}


\section{Student Honor Code \& Academic Misconduct}

Mississippi State has an approved Honor Code that applies to all students. The code is as follows: 
\begin{quote}
\vspace{-2mm}
As a Mississippi State University student, I will conduct myself with honor and integrity at all times. I will not lie, cheat, or steal, nor will I accept the actions of those who do.
\vspace{-2mm}
\end{quote}
Upon accepting admission to Mississippi State University, a student immediately assumes a commitment to uphold the Honor Code, to accept responsibility for learning, and to follow the philosophy and rules of the Honor Code. Student will be required to state their commitment on examinations, research papers, and other academic work. Ignorance of the rules does not exclude any member of the MSU community from the requirements or the processes of the Honor Code. For additional information, please visit: \url{http://honorcode.msstate.edu/policy} and \url{http://students.msstate.edu/studentconduct/}.

\textbf{To be clear, students who cheat in any way will be penalized. Cheating includes giving as well as receiving help when such help is not explicitly allowed.} Plagiarism is also a form of cheating. The best way to avoid anything that might be academic misconduct is to put yourself in a position where you don't need to cheat or plagiarize. Don't get behind, and if there are things that you don't understand, give yourself time to figure them out or ask me about them.

If you have any further questions about what constitutes cheating, either ask me or see the University’s policy on academic dishonesty. (I am happy to answer any questions about what is and is not allowed. But ask me before you do something questionable.)





\section{Coursework \& Grading}

Letter grades will be assigned using the standard Mississippi State scale (an A is 90 – 100 percent, a B is 80 – 89 percent, a C is 70 – 79 percent, etc.). The grades will be set based on this coursework and these percentages:
\begin{description}
\item quizzes: 15 percent
\item homework \& discussion assignments: 45 percent
\item three papers: 40 percent
\end{description}
The quizzes will consist of multiple choice questions, and they are open-book assignments
(although you have to take them alone). Generally, there will be one or two quizzes per week.
In addition to the quizzes, most weeks there will be a homework or discussion assignment. 
Three longer writing assignments (about 2 to 4 pages) will be due on these dates:
\begin{description}
\item paper 1: Sunday, February 14
\item paper 2: Sunday, March 21
\item paper 3: Sunday, April 25
\end{description}
Every due date is firm. Quizzes, homework assignments, and discussion assignments cannot be completed after their due dates. Papers will be accepted late, but late paper will be penalized. Briefly, the reasons for this are as follows. First, although different professors have different policies, when you sign up for a course, you are agreeing to work within the schedule set for that course. If you can’t do that—for whatever reason (good or bad)—you should expect that your grade will be affected. Second, it is necessary
to keep the entire class organized and moving in the right direction. For large classes especially, it’s not possible to do that and have flexible due dates.



%
\section{MSU Statements on COVID-19}

\subsection{MSU facial covering policy} 

To safeguard the health of all members of the MSU campus during this global pandemic, the university has reconfigured classroom spaces and adjusted room capacities to assure adequate physical distance between all individuals in each room.  In addition, the university has published requirements for the use of face coverings for everyone on campus, including specific requirements for their use in all classrooms, labs, and shared office spaces regardless of physical distancing.  In order to mutually protect the students’ freedom to learn and the instructor’s ability to teach in a safe classroom environment, everyone in this classroom is required to wear a face covering in the classroom in accordance with MSU policy (\href{https://www.msstate.edu/covid19/return-plan}{\textit{Comprehensive Health \& Safety Return Plan}}).  If a student cannot wear a face covering due to a medical condition, he or she should request an accommodation via the Office of Disability Support Services.  \textcolor{red1}{If a student simply doesn’t want to wear a face covering, he or she will not be permitted to remain in the classroom or lab.}


\subsection{Course continuity in the event of an online migration} 

In the event that face-to-face classes are suspended due to the pandemic or its effects, the instructor will continue instruction in a manner that best supports the course content and student engagement. In this event, all instructors will notify all students of the change via their university email address (the official vehicle for communication with students).  At that time, they will provide details about how instruction and communication will continue, how academic integrity will be ensured, and what students may expect during the time that face-to-face classes are suspended. If a student becomes unable to continue class participation, the student should contact their instructor and advisor for guidance.



\section{COVID-19}

Taking the necessary precautions to safeguard the health of all members of the MSU community is important but will not specifically apply to this course since it is online. Nonetheless, you should be familar with MSU's current policies, which are explained here: \url{https://www.msstate.edu/covid19/return-plan}.

%{\textit{Comprehensive Health \& Safety Return Plan}}. 


% \href{https://www.msstate.edu/sites/www.msstate.edu/files/SafeReturnBooklet.pdf}{\textit{Safe Return Booklet}}.

\section{Support Services}

It is the policy of Mississippi State University to accommodate students with special needs and learning disabilities as per the MSU Student Support Services policy. Students seeking accommodations on the basis of a disability or special need must contact the Office of Student Support Services (Room 01 Montgomery Hall, 662-325-3335) to verify eligibility. For more information, please visit \url{http://www.sss.msstate.edu/}.

Academic accommodations and services are based on an individual’s needs.  All documentation is confidential. 


\section{Title IX}

MSU is committed to complying with Title IX, a federal law that prohibits discrimination, including violence and harassment, based on sex.  This means that MSU’s educational programs and activities must be free from sex discrimination, sexual harassment, and other forms of sexual misconduct.  If you or someone you know has experienced sex discrimination, sexual violence, or harassment by any member of the University community, you are encouraged to report the conduct to MSU’s Director of Title IX/EEO Programs at 662-325-8124 or by e-mail to \url{titleix@msstate.edu}. Additional resources are available here:
\begin{description}
\item \url{https://www.oci.msstate.edu/focus-areas/title-ix-sexual-misconduct/}
\item \url{http://students.msstate.edu/sexualmisconduct/}  
\item \url{http://www.msstate.edu/web/security}
\end{description}

\section{University Safety Statement}

Mississippi State University values the safety of all campus community members. Students are encouraged to register for Maroon Alert texts and to download the Everbridge App. Visit the Personal Information section in Banner on your MyState portal to register. 

To report suspicious activity or to request a courtesy escort via Safe Walk, call University Police at 601-934-0863, or in case of emergency, call 911. For more information regarding safety and to view available training resources, including helpful videos, visit: \url{http://ready.msstate.edu}.



\end{document}