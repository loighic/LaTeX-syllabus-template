%%%
%%%  Use XeLaTex !!!!!

\documentclass[11pt,oneside]{article}
\usepackage{syllabus}

% local definitions
\def\mydepartment{Division of Arts \& Sciences}
\def\myaddress{071 College Park}

\def\mytitle{\MakeUppercase{\Huge{Introduction to Philosophy}}}	
\def\subtitle{\footnotesize{PHI 1103 (201)}}
\def\meeting{Available meeting times outside of class: Tuesday, 1:00 - 4:00 pm}
\def\semester{\footnotesize{Spring 2020}}


\begin{document}

\vspace*{.75cm}
\begin{flushleft}
\textsf{\subtitle~~$\cdot$~~\semester\\
\vspace{.75cm}
%\subtitle\\
\vspace{3mm}
\mytitle\\
\vspace{5mm}
\normalsize{\myauthor}\\
\myemail\\}
%\myweb\\}
\vspace{1.5mm}
%\myphone~$\cdot$~\myaddress~$\cdot$~\meeting\\}
\end{flushleft}
\vspace{15mm}

\section{Preliminaries}

\noindent Email is the best way to reach me, and I will usually respond within 24 hours. If you would like a more immediate response, you can try calling or sending a text message to 267-416-0292. But don’t leave a voicemail. I won’t get it.
 
\vspace{2mm}
\noindent \meeting ~(\myaddress)

%With suitable notice, I may make any changes to the syllabus, readings, assignments, or schedule at any time during the semester.



\section{Readings}

All of the readings for this course are posted on Canvas. Print, read, and bring to class the reading or readings assigned for that class meeting.


\section{Course Description}

Before taking a philosophy course, most people are unfamiliar with the subject. So, by way of introduction, here is an excerpt from a lecture on Plato’s dialogue the \textit{Apology} (which we’ll read) by the philosopher Steven Smith, 
\begin{quote}
Philosophy grows out of a desire to replace opinion with knowledge, to replace opinion or belief with reason. For philosophy, it is not enough simply to hold a belief on faith, but one must be able to give a rational account, a reasoned account for one’s belief. Its goal, again, is to replace civic faith with rational knowledge. And, therefore, philosophy is necessarily at odds with belief and with this kind of civic faith. … The philosopher seeks to judge those beliefs in the light of true standards, in the light of what is always and everywhere true, as a quest for knowledge. There is a necessary and inevitable tension between philosophy and mere belief.
\end{quote}

Philosophy is different than science, but only to a degree. Science investigates the world by collecting data and doing experiments. Philosophy, meanwhile, usually does neither of these things. But still, the philosopher seeks knowledge, and the tool for seeking knowledge is reason. Given some facts (perhaps uncovered by science), what follows? What is likely to be---or must be---true? \ldots


\section{Learning Objectives}

Naturally, one objective is to become acquainted with the assigned material. Other, no less important, goals are improving your reading comprehension, critical thinking, and writing skills.

\section{Coursework \& Grading}

Letter grades will be assigned using the standard Mississippi State scale (an A is 90 – 100 percent, a B is 80 – 89 percent, a C is 70 – 79 percent, etc.). The grades will be set based on this coursework and these percentages:
\begin{description}
\item quizzes \& assignments: 30 percent
\item paper: 15 percent
\item three tests: 55 percent
\end{description}

A 5 to 6 page paper will be due near the end of the semester. . . .


\section{Student Honor Code \& Academic Misconduct}

Mississippi State has an approved Honor Code that applies to all students. The code is as follows: 
\begin{quote}
\vspace{-2mm}
\textcolor{red1}{As a Mississippi State University student, I will conduct myself with honor and integrity at all times. I will not lie, cheat, or steal, nor will I accept the actions of those who do.}
\vspace{-2mm}
\end{quote}
Upon accepting admission to Mississippi State University, a student immediately assumes a commitment to uphold the Honor Code, to accept responsibility for learning, and to follow the philosophy and rules of the Honor Code. Student will be required to state their commitment on examinations, research papers, and other academic work. Ignorance of the rules does not exclude any member of the MSU community from the requirements or the processes of the Honor Code. For additional information, please visit: \url{http://honorcode.msstate.edu/policy} and \url{http://students.msstate.edu/studentconduct/}.

\textbf{To be clear, students who cheat in any way will be penalized. Cheating includes giving as well as receiving help when such help is not explicitly allowed.} Plagiarism is also a form of cheating. The best way to avoid anything that might be academic misconduct is to put yourself in a position where you don't need to cheat or plagiarize. Don't get behind, and if there are things that you don't understand, give yourself time to figure them out or ask me about them.

If you have any further questions about what constitutes cheating, either ask me or see the University’s policy on academic dishonesty. (I am happy to answer to any questions about what is and is not allowed. But ask me before you do something questionable.)





\section{MSU Statements on COVID-19}

To safeguard the health of all members of the MSU campus during this global pandemic, the university has reconfigured classroom spaces and adjusted room capacities to assure adequate physical distance between all individuals in each room.  In addition, the university has published requirements for the use of face coverings for everyone on campus, including specific requirements for their use in all classrooms, labs, and shared office spaces regardless of physical distancing.  In order to mutually protect the students’ freedom to learn and the instructor’s ability to teach in a safe classroom environment, everyone in this classroom is required to wear a face covering in the classroom in accordance with MSU policy (\url{https://www.msstate.edu/sites/www.msstate.edu/files/SafeReturnBooklet.pdf}).  If a student cannot wear a face covering due to a medical condition, they should request an accommodation via the Office of Disability Support Services.  \textbf{If a student simply doesn’t want to wear a face covering, they will not be permitted to remain in the classroom or lab.}

In the event that face-to-face classes are suspended due to the pandemic or its effects, the instructor will continue instruction in a manner that best supports the course content and student engagement. In this event, all instructors will notify all students of the change via their university email address (the official vehicle for communication with students).  At that time, they will provide details about how instruction and communication will continue, how academic integrity will be ensured, and what students may expect during the time that face-to-face classes are suspended. If a student becomes unable to continue class participation, the student should contact their instructor and advisor for guidance.


\section{Two Other Rules}

\begin{description}
\item[(1)] Cell phone use, including texting, is not allowed in class. 
\item[(2)] You may not use a laptop or tablet in class. 
\end{description}

\noindent You may, at the very beginning of class, put upcoming assignments into your phone. But besides that exception, I should never see a cell phone or an open laptop during class.
These rules are designed to benefit you. Although you (sitting in class) may not perceive texting as rude or distracting, the person who is addressing you will usually interpret it negatively. This applies, not only to professors, but to the people you will work with and work for once you graduate. Learning to manage when you look at and use your phone is a good habit to develop now.

Laptop use can also, in some situations, be perceived as rude, and often it is obvious that the laptop user is not just taking notes. But more importantly, using a laptop creates an enormous barrier to paying attention. Every student who uses a laptop in the classroom spends time on things that are not related to class (email, Facebook, etc.). Consequently, these students get little or no benefit from being in the classroom, and the students’ grades indicate as much. (And don’t think that you’re good at ``multi-tasking.'' You’re not. The human brain doesn’t work that way.)


\section{Schedule}

\begin{minipage}{\textwidth}
See the calendar on Canvas for the exact schedule. This is only an outline.

\vspace{3mm}
\begin{tabular}{@{}ll@{}}
\midrule
Weeks 1 \& 2 & An introduction to arguments (Johnson)\\
Week 3 & Copernicus and Galileo (Gingerich)\\ 
Weeks 4 \& 5 & What is the mind and who has one? (Johnson)\\
Week 6 & \textbf{test 1}\\
Week 7 & The severely demented, minimally functional patient (Arras)\\
Weeks 8 \& 9 & Free will (Johnson and Ayer)\\ 
Week 10 & \textbf{test 2}\\
Week 11 & Are we living in a computer simulation? (Bostrom)\\
Week 12 & God and the problem of evil (Rowe)\\
Weeks 13 \& 14 & Apology (Plato)\\
Week 15 & Crito (Plato)\\
Exam Week & \textbf{test 3}\\
\midrule
\end{tabular}
\vspace{3mm}
\end{minipage}

\section{Academic Tutoring \& Writing Center}

The Academic Tutoring and Writing Center is available to help students with any part of the writing process, from understanding the assignment to the finished product. The ATWC mainly operates virtually through Canvas, but it is also available for in-person meetings at the College Park campus on Wednesdays from 3:30 – 5:30 pm (no appointment needed). For more information, contact Leslie Pevey (\url{atwc@meridian.msstate.edu}; or, on Wednesdays between 3:30 and 5:30 pm, you can call 601-484-0248).


\section{Support Services}

Students who need academic accommodations based on a disability should visit the Student Support Services located on the upper level of the College Park campus and speak with Ms. Amy Smith (\url{asmith@meridian.msstate.edu}, 601-484-0234). This should be done before any class assignments are due. Further information about academic accommodations as well as other student services is available here: \url{http://www.meridian.msstate.edu/student-services}.

Academic accommodations and services are based on an individual’s needs.  All documentation is confidential. 


\section{Title IX}

MSU is committed to complying with Title IX, a federal law that prohibits discrimination, including violence and harassment, based on sex.  This means that MSU’s educational programs and activities must be free from sex discrimination, sexual harassment, and other forms of sexual misconduct.  If you or someone you know has experienced sex discrimination, sexual violence, or harassment by any member of the University community, you are encouraged to report the conduct to MSU’s Director of Title IX/EEO Programs at 662-325-8124 or by e-mail to \url{titleix@msstate.edu}. Additional resources are available here:
\begin{description}
\item \url{http://students.msstate.edu/sexualmisconduct/}  
\item \url{http://www.msstate.edu/dept/audit/PDF/91118.pdf} 
\item \url{http://www.msstate.edu/web/security}
\end{description}

\section{University Safety Statement}

Mississippi State University values the safety of all campus community members. Students are encouraged to register for Maroon Alert texts and to download the Everbridge App. Visit the Personal Information section in Banner on your MyState portal to register. 

To report suspicious activity or to request a courtesy escort via Safe Walk, call University Police at 601-934-0863, or in case of emergency, call 911. For more information regarding safety and to view available training resources, including helpful videos, visit: \url{http://ready.msstate.edu}.



\end{document}